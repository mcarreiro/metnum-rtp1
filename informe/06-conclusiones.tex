\section{Conclusiones}

\subsection{Random vs Greedy}

Como observamos en los resultados para el caso en el que la sanguijuela/s se encuentran en el medio claramente es mejor el greedy, ya que siempre mata a la sanguijuela correcta a diferencia del random que sólo lo hizo en uno de los 3 intentos. Por otro lado si tenemos en cuenta los tiempos que pueden llegar a demorar recalcular las temperaturas, esto le da todavìa una mayor importancia, ya que se reduce considerablemente el tiempo de ejecución.

Por otro lado en el caso en el que la sangijuela a sacar no era la más cercana, el random en promedio obtuvo mejores resultados. Sin embargo al aumentarle la granularidad esto se revertió nuevamente debido a que aumentó la temperatura del PC y esto provocó que si o si fuese necesario matar por lo menos dos sanguijuelas, entre ellas la mas cercana. \\
Lo que pensamos con respecto a esto es que hay casos bastantes puntuales en los que el greedy falla y el random podría devolver una mejor solución, aunque esto también depende del porcentaje de sanguijuelas que se puedan sacar para solucionar el problema. Es decir, si de 5 sanguijuelas por ejemplo la solución no está en sacar la mas cercana sino cualquiera de las otras 4, el random sería el mejor, pero si la correcta fuera una sola, el random volvería a ser tan o mas ineficiente que el greedy ya que tendría una probabilidad de 1/5 de elegir la correcta la primera vez.\\
Osea, el random se comportaría mejor que el greedy sólo en el caso de que la sanguijuela a sacar no sea la más cercana y haya otras n/2 (o más) posibles para sacar de forma que resuelvan el problema.

Finalmente conlcuimos que los casos en los que el random sería mejor que el greedy son tan puntuales y nuestro problema, en principio, es tan genérico que son despreciables. Por lo que el greedy termina siendo claramente mejor.






