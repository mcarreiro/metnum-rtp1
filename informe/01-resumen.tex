\section{Resumen}

Mediante el manejo de matrices se buscará modelar y resolver un problema de la realidad. Cómo la realidad está compuesta de infinitas variables dicha modelización implicará una inevitable discretización. Es decir trabajar con una 
cantidad acotada de variables del problema (sólo las relevantes).

El problema en cuestión es representar un parabrisas al cual se le adhieren sanguijuelas que le producen calor en un radio de las mismas. Queremos evitar que el parabrisas llegue a un punto crítico, que implicaría la destrucción de este. El punto crítico ocurre cuando el punto (los puntos) del centro tienen más de 250 grados. Para evitar que llegue al punto crìtico debemos deshacernos de las sanguijuelas pero no tenemos la energía para destruirlas a todas, por lo que tenemos que achicar la cantidad de sanguijuelas a eliminar de forma tal que el parabrisas no se rompa.

La modelización del mismo no es trivial. Es más, podría decirse que este proceso es mucho más complejo y costoso que la solución en sí. ¿Por qué? porque la creación de la matriz que represente al parabrisas y el cálculo de las temperaturas en cada punto, si no se usa buen método, podría llegar a demorar mucho tiempo.

Por esto es que a lo largo de este tp haremos mucho foco en como calcular las temperaturas, como optimizar el espacio ocupado por la matriz obtenida y como optimizar lo más posible todas las operaciones matriciales.


