\section{Resumen}

En el presente informe se busca mostrar como modelar y resolver un problema de la realidad utilizando matrices. Como la realidad está compuesta de infinitas variables dicha modelización implicará una inevitable discretización. Es decir trabajar con una 
cantidad acotada de variables del problema (sólo las relevantes).

El problema se enfoca en la representación de un parabrisas al cual se le adhieren sanguijuelas, produciendo calor con la superficie de su cuando de determinado radio Para evitar la destrucción del parabrisas, debemos evitar que llegue a un punto crítico, el cual ocurre cuando el punto de centro tiene más de 235${}^o$C. Para ello, debemos deshacernos de las sanguijuelas, pero al no tener la energía suficiente para destruirlas en su totalidad, debemos disminuir la cantidad de sanguijuelas en el parabrisas y lograr eliminarlas para que el mismo no se rompa.

La modelización del mismo no es trivial. Es más, podría decirse que este proceso es mucho más complejo y costoso que la solución en sí. ¿Por qué? porque la creación de la matriz que represente al parabrisas y el cálculo de las temperaturas en cada punto, si no se usa un buen método, podría llegar a demorar mucho tiempo. 

Por esto es que a lo largo del TP haremos mucho foco en como representar en forma de matriz el parabrisas, como este puede llegar a tener propiedades que utilizaremos para ahorrar el espacio ocupado. Una vez que tengamos esto, necesitamos saber si realmente estamos bajo un punto crítico. Esto estará determinado por la posición central  de la matriz que discretiza el parabrisas $(altura/2,ancho/2)$ (donde altura y ancho son las cantidad de celdas correspondientes) y su temperatura (si es mayor o menor a 235${}^o$C) para saber así si necesitamos deshacernos de las sanguijuelas para poder seguir navegando.

Una vez que tengamos esto plantearemos casos de test interesantes para analizar el tiempo de decisión de una situación de peligro y que el método elegido para deshacernos de las sanguijuelas nos minimiza la cantidad a destruir.

\textit{Se acompaña el informe con el código en C++ y los archivos .in que utilizamos para la experimentación}


